\documentclass[11pt]{article}
\usepackage[sc]{mathpazo} %Like Palatino with extensive math support
\usepackage{fullpage}
\usepackage[authoryear,sectionbib]{natbib}
\linespread{1.7}
\usepackage[utf8]{inputenc}
\usepackage{lineno}
\usepackage{amsmath}
\usepackage{graphicx}

\begin{document}

\noindent
\textbf{Title}: 

\bigskip

\noindent
\textbf{Running title}: Managing a macroparasite: leveraging data and models to mitigate human risk of raccoon roundworm infection [too specific?]

\bigskip

\noindent
\textbf{Authors}: Sara B. Weinstein$^1$ and Mark Q. Wilber$^2$

\bigskip

\noindent
\textbf{Author affiliations}: \\
1. \\
2. Ecology, Evolution and Marine Biology, University of California, Santa Barbara, Santa Barbara, CA, 93106 \\

\bigskip

\noindent
\textbf{Corresponding author}:

\bigskip

\noindent
\textbf{Word Count}: 

\bigskip

\noindent
\textbf{Keywords}: Individual Based Model, Raccoon, Macroparasite,
Wildlife disease, Zoonosis, Parasite, Management\ldots{}..

\bigskip

\noindent
\textbf{Number of figures}:  \\
\textbf{Article type}: 

\clearpage

\section{Introduction}

Rinderpest is eliminated \citep{Roeder2011}, rabies is reduced \citep{Freuling2013}, and strategies are currently being implemented to manage infectious disease in a number of other wildlife systems [CWD, Koala STDs, etc.]. In many cases, these management strategies have been
guided by mathematical models \citep[e.g.][]{Restif2012,McCallum2017}.  Mathematical models are an important component when managing wildlife disease because the can be used to inform field-based management \emph{a priori} and assess the efficacy of a management strategy \emph{post-hoc} \citep{Restif2012}.  While number of models have been successfully applied to managing
viral, bacterial and protozoan agents in wildlife populations [give examples, many found in McCallum], models directed
towards managing wildlife macroparasites are still lacking \citep[][, but note significant work has been done in managing macroparasites in livestock populations (citaitons)]{McCallum2017}.

The scarcity of management-based models for macroparasites in wildlife is likely driven by two reasons. First, macroparasites, such as helminths that do no directly reproduce within a host \citep{AndersonandMay1979}, often have complex life cycles and tend to have load-dependent effects on host fecundity and mortality. While these characteristics by no means preclude modeling of macroparasites in wildlife, they do require models that track the distribution of parasite loads across hosts in addition to the total host population and mean parasite load \citep{AndersonandMay1978}.  These types of models can quickly become challenging to implement, parameterize, and analyze when they are built to ask system-specific questions regarding parasite management in wildlife systems \citep{McCallum2017}.  Second, in contrast the dramatic effects that microparasite epidemics can have on wildlife populations (Prairie dog epidemics, the mongollian herd epidemic, sea star wasting disease, avian bird flue, etc.), the role of macroparasites as a general regulating factor of wildlife populations is unclear \citep{Tompkins2002,Tompkins2011}. This has potentially made the perceived need for detailed macroparasite models in managed wildlife systems less critical. [but see X and X for some important exceptions...I don't like this sentence]. However, macroparasites that have equivocal effects on the dynamics of wildlife populations can still be an important concern for human health (Page, Graeff, Kazacos).  To manage this risk to human health, macroparasite models are needed that simultaneously incorporate the macroparasite dynamics in a wildlife reservoir as well as the the risk this reservoir poses to human populations.  

The macroparasite raccoon roundworm is increasingly recognized as a threat to both human
and wildlife health \citep{Page2011,Weinstein2017}, but, like many
macroparasites, is difficult to control due to its complex life cycle
and resistant environmental infectious stages. Worms mature in the
raccoon gut and infected raccoons can release over a million parasite
eggs per day [citation]. Eggs survive for over a year (Shafir) and accumulate at
communal raccoon defecation sites termed latrines [citation]. Individual
raccoons contribute to multiple latrines and these contaminated sites
expose raccoons and other species to eggs. Eggs infect juvenile
raccoons, leading to high parasite loads in animals as young as four
months \citep{Weinstein2016} [TODO: Fill in these citations: Kazacos, Boyce]. Raccoon susceptibility to eggs
declines with age and adult raccoons acquire worms by eating infected
small mammals and birds [citation]. In these birds and mammals, larval worms do
not mature. Instead, larval worms migrate from the gut into other
tissues including the brain, often causing fatal neurological damage.
Raccoons scavenge these incapacitated animals [citation], and this trophic
transmission maintains infection in older raccoons that are less
susceptible to eggs [citation]. Nearly all infected raccoons release eggs;
however higher parasite loads in juveniles suggests that this age class
contributes disproportionately to human disease risk [citation].

Human risk might be reduced by removing \emph{B. procyonis} eggs from
the environment, treating infected raccoons, or reducing raccoon
populations. Culling is often considered as a possible control
disease control strategy \citep{Wasserberg2009,Langwig2015}. However, although a common component of
rabies control \citep{Rosatte1986}, culling is controversial and not
always effective \citep{Choisy2006a,Beeton2011,Morters2013}. Birth control is a less controversial method to
reduce wildlife populations \citep{Smith2002a}, but is often slower, more
expensive, and less effective than culling (citations). As an alternative to
reducing host populations, parasite populations can be reduced through
mass drug administration (Qureshi 1994 et al, deer fascioloides) or
vaccination (anthrax antelope De vos 1973 Koedoe, bison; elk brucellosis
(rabies). Although no \emph{Baylisascaris} vaccines exist, the mass
baiting methods developed for rabies vaccines can also deliver deworming
medication (anthelminthics) \citep{Smyser2015} [What about Smyser 2013?]. These anthelminthic
baits can reduce \emph{B. procyonis} prevalence in wild raccoons, but
raccoons are rapidly re-infected unless latrines are also removed \citep{Page2011} [citation for Page et al. 2014?]. Latrine, parasite, and raccoon removal
all could reduce human \emph{B. procyonis} exposure; however,
experiments that compare these strategies are difficult to implement in
wild animal populations.

In this study we use a combination of previously published results, field estimated parameters, and model fitting approaches to parameterize an individual-based model of raccoon roundworm.  We use this model to investigate how different raccoon roundworm management strategies affect human risk of \emph{B. procyonis} infection. As in many models of infectious disease, the structure of parasite transmission is one of the most important, yet challenging, portions of the host-parasite interaction to estimate.  Combining age-intensity with Approximate Bayesian Computing (ABC), we attempt to infer certain characteristics of the transmission function in raccoon roundworm systems.  Given that age-intensity and -prevalence data are commonly collected in parasitological studies, our approach highlights that, given some prior knowledge of the basic biology driving a host-parasite system, this type of data can be used to make inference about the transmission process [WC: Inference isn't really the right word...]. [GENERAL STATEMENT]

\section{Methods}

\subsection{Model Description}

To evaluate raccoon roundworm management strategies, we built an
individual-based model for a macroparasite with long-live
environmental infectious stages and complex life cycle. We describe this
model using the Overview, Design concepts, Details (ODD) protocol \citep{Grimm2006,Grimm2010}, providing first an Overview of the model, then
describing the Design Concepts, followed by the Details of submodels,
parameterization and initialization. Additional details on submodels and
annotated code are available in the Supplementary Material.

\subsection{Overview}

\subsubsection{Purpose}

We developed an individual-based macroparasite model for raccoon and
raccoon roundworm to compare how host, parasite, and infective stage
targeted management affect host demography, parasite demography and
human exposure.

\subsubsection{Entities, state variables, and scales}

In this model, raccoon are the main model entity and exist as an age
structured, closed, all-female population. Five attributes describe each raccoon: an identification number, an age, an alive
or dead status, a location and a parasite load. We track the female parasite
load (i.e. the infrapopulation) within each raccoon. Raccoons can acquire new
worms at each time step and, for each raccoon, we explicitly track the age of each
worm cohort. Worms accumulate and die over time, and we
calculate parasite load by summing across parasite cohorts within each raccoon.
Infected raccoons contaminate the environment with long-lived parasite
eggs, which we model as an environmental egg pool. As even a single worm
can produce more than a million eggs per month [citation], we assume that infected
raccoons rapidly saturate local latrines with eggs. Thus, we do not
explicitly track each egg, and instead use infection prevalence in the
local raccoon population as a proxy for local egg production. Eggs decay
over time () and we calculate the current environmental egg pool as a
weighted sum of past egg production. Eggs also infect rodents that co-occur with raccoons. We use the
environmental egg pool to estimate rodent infection at each time step. [However, because \emph{B. procyonis} is not reproductive in rodents, infected rodents do not directly contribute to the environmental egg pool.]

We track hosts, parasites, and environmental egg pools in a spatially
implicit 20 km\textsuperscript{2} world with 10 equally sized 2
km\textsuperscript{2} zones. Human population density increases
consecutively from zones one through ten (Fig X, x) and zones with
higher human density support larger raccoon populations (). We track the
raccoon population, parasite population, and egg pool in each zone at
monthly time steps for 400 months, using the egg pool in each zone to
calculate total human risk within the simulated world.

\subsubsection{Process overview and scheduling}

We update the model on a monthly time step. The first event that occurs in a time step is raccoon survival. Surviving raccoons then reproduce, lose
parasites, gain parasites, disperse, age, and contribute to the
environmental egg pool. Following the typical raccoon life cycle,
raccoons reproduce once per year, with kits entering the population in
the following time step. All other events occur monthly in all ten
zones.

\subsection{Design Concepts}

\subsubsection{Emergence and Interactions}

Although some infectious agents can regulate their hosts (), raccoon
roundworm does not regulate raccoons [citations]. Unmanaged raccoon populations are
limited by resource availability (), which we model by setting a raccoon
carrying capacity in each zone. Following the Ricker function (Gurney
and Nisbet, Hastings et al 2012), as raccoon population density
increases within a zone, reproduction declines. Simulated raccoons
reproduce once per year (in the spring), and this pulsed reproduction
produces realistic annual population cycles and age distributions for an
unmanaged raccoon population (gehrt, others).

We add macroparasites to this simulated raccoon population following
classic macroparasite modeling theory (eg. Anderson and May, Cornell,
McCallum). Individual host-parasite interactions generate an aggregated
parasite population with realistic age-intensity and age-prevalence
profiles. The parasite population cycles due to
changing host demography, and these emergent patterns match the widely
observed fall rise in mean raccoon parasite loads (eg page 2016,
weinstein2016, many others).

\subsubsection{Stochasticity}

We determine the following modeling events using a random number generator in R version X: raccoon survival (0=death, 1=alive), raccoon reproduction (0=no reproduction, 1=reproduction), raccoon litter size (0 [CHECK], 1, 2 kits), raccoon dispersal (CHECK), parasite loss (), and parasite gain ().  Each of these probability of a particular realization of any of these events depends on the biology of the raccoon-\emph{B. procyonis} described in \emph{Submodels}.

\subsubsection{Initialization}

We began each simulation with 500 two-year old raccoons stochastically
distributed throughout the 20 km\textsuperscript{2} world according to
zone specific carrying capacities (average starting zone density: ).
Initially, each raccoon hosted 10 worms and no eggs contaminated any
zones. The mean rodent worm load was initially 3.49 based on
empirical observations (X) and the variance of rodent worms
was set to 87.08 based on the canonical scaling relationship between
log-mean parasite load and log variance in parasite load \citep{Shaw1995}.

\subsection{Submodels}

\subsubsection{Raccoon death}

We assume that all raccoons experience a constant per month death
probability and then augment this age-independent death rate to account
for high nestling mortality (Montgomery 1969) and senescence. Nestling
raccoons (\textless{}1 month) have constant added risk of death and we
model senescence as an increasing age-specific hazard rate,
parameterized to ensure that raccoons cannot live past 20 years old.
These three mortality processes lead to a ``bathtub''-shaped hazard
function for raccoon mortality (Fig. X)

Raccoons can also die from \emph{B. procyonis} infection if worms cause
an intestinal obstruction (Stone, 1983; Carlson and Nielsen, 1984).
Fatal obstructions can occur at loads as low as 141 worms and have been
recorded in animals with 636 and 1,321 worms (Stone, 1983; Carlson and
Nielsen, 1984), however raccoons can survive with over 200 worms
(Kazacos 2001). To model this potential parasite-induced host mortality,
we used the logistic parasite-induced mortality function
\(1 - \frac{e^{\beta + \alpha*log(load + 1)}}{1 + e^{\beta + \alpha*log(load + 1)}}\)
, where \textbackslash{}alpha gives the per parasite effect on the log
odds raccoon survival probability and \textbackslash{}beta roughly
corresponds to the threshold at which parasite-induced mortality begins
to occur \citep{Wilber2016}. We estimated these parameters using the aforementioned data [which data]. Our parameterization corresponds with the empirical observation that parasite-induced mortality is not a large factor affecting raccoon dynamics.

\subsubsection{Raccoon reproduction}

Raccoons typically reproduce once a year and can have up to 2 female
young per litter (Gehrt 2001, Fritzell 1985, Cowan 1973, clark et al
1989). Although there is some evidence that populations in the far south
of the range may breed year round (Troyer et al 2014) and late litters
do occur (Gehrt and Fritzell 1996, 1978a?), here we assume that all
reproduction occurs in a single spring pulse.

The model assumes that average raccoon litter size in a given year
decreases with increasing raccoon density. This density-dependence
corresponds with reproduction in dense populations being limited by both
the availability of nest sites and increased infanticide risk in dense
populations (Gehrt and Fritzell 1999; hauver 2010; Wolff 1997).
Specifically, we model this density-dependence following a Ricker
function (Table X).

\subsubsection{Worm transmission}

Raccoons acquire \emph{B. procyonis} through two transmission routes:
encounter with \emph{B. procyonis} eggs in the environment and by
consuming other animals already infected with \emph{B. procyonis} adults
(e.g. rodents) (citations).  We assume that raccoons encounter eggs with some probability that is proportional to the total number of eggs in the environment (on a zone-dependent basis). Conditional on encountering eggs, the total number of eggs encountered follows a negative binomial distribution with a fixed aggregation parameter and fixed number of mean eggs encountered.  Finally, the probability that any one of these eggs actually infects the raccoon is a decreasing function of how infected the raccoon (i.e. increased load decreases susceptibility) and is zero if the raccoon is older than 4 months (citations). Finally, the worms are acquired from rodents if the raccoon is older than 4 months.  Rodents are encountered and consumed with some probability and the number of worms in the rodent follows a negative binomial distribution with the mean that has a maximum value of 3.49 worms per mouse and decreases with decreasing environmental egg pool. As stated above, variance scales with the mean according to Taylor's Power Law \citep{Shaw1995}.  The aggregation parameter and mean parameter when encountering eggs in the environment, the parameter the determines how encounter probability scales with the size of the environmental egg pool, and the rodent encounter probability are all estimated using the ABC method described below.

\subsubsection{Worm death}

Worms die according to an age-dependent survival function that follows the logistic curve $\frac{1}{\exp(-(a + b \text{age})}$.  This means that in any time step, worms of age \emph{age} have a probability of dying according to the above function (Fig X).  This function was parameterized with data from Olsen 1958 [Sara, more detail about this study here].  Note that we do not include any negative density-dependence in parasite death rate as we instead included this density-dependence in the probability of a worm successfully establishing (see \emph{Worm transmission}).

\subsection{Fitting susceptibility and transmission parameters with ABC-SMC}

As described in Table X, most of the parameters in this model were
estimated based on previous \emph{B. procyonis} studies in other systems
and previous work by S.W. in the current system. However, there were
five transmission and susceptibility parameters in the model that were
largely unknown in the raccoon-\emph{B. procyonis} system. These were
the mean number of eggs encountered by a raccoon in the environment over
a monthly time step, the variability in the number of eggs encountered
over a monthly time step, an egg encounter parameter relating the eggs
in the environment to the probability of a raccoon encountering an egg,
the probability of a raccoon consuming a rodent in a month, and a
parameter determining how concomitant immunity increases with raccoon
age (Table X).

To estimate these parameters, we used Approximate Bayesian Computing
with Sequential Monte Carlo (ABC-SMC) \citep{Sisson2009,Beaumont2010,Kosmala2015}. ABC is typically referred to as a likelihood
free parameter estimation method and is often used with simulation-based
models in which the likelihood of a data point given the model is
intractable to compute. Instead, ABC uses a user-defined distance
function to determine how ``well'' the simulation model parameterized
with a set of parameters drawn from some prior distribution can
reproduce the observed data. This approach allows one to estimate an
approximate posterior distribution for unknown parameters in a model based
on how well they can reproduce observed data.

To use ABC to parameterize the unknown parameters in the
raccoon-\emph{B. procyonis} IBM, we used observed age-abundance and
prevalence profiles for raccoons and \emph{B. procyonis} in our system
of interest \citep{Weinstein2016}. For many mammal macroparasite
systems, this type of intensity and prevalence data is the best
available parasitological data as longitudinal, individual-based
measures of worm intensity are nearly impossible due to both the
difficulty of re-trapping individual hosts and the destructive nature of
measuring parasite loads.
Therefore, developing methods to estimate transmission and
susceptibility parameters from this type of data is important for a
wide-range of host-macroparasite systems.

Using this age-intensity/prevalence data for the raccoon-\emph{B.
procyonis} system, we then implemented the ABC-SMC algorithm as follows \citep{Sisson2009,Toni2009}



\begin{enumerate}
\def\labelenumi{\arabic{enumi}.}
\item
  We specified uniform priors around our five unknown transmission and
  susceptibility parameters: {[}PRIORS HERE{]}
\item
  We drew 10,000 samples from each prior distribution such that we had
  10,000 vectors of parameters (in the ABC literature each parameter
  vector is called a particle).
\item
  For each of the 10,000 particles, we simulated our model for 100
  months to remove the effects of the initial conditions.
\item
  For a given simulation, we then sampled 189 raccoons over the last 24
  months of the simulation which is consistent with the number of
  raccoons used to build the age- abundance/prevalence profiles given in \cite{Weinstein2016}. Specifically, we distributed these 189 sampled
  raccoons according the eight age classes given in \cite{Weinstein2016}.
\item
  Using this sample, we then constructed the age-abundance/prevalence
  profile from the simulated data and compared it to the observed age-
  abundance/prevalence.
\item
  To compare the simulated and observed data, we first combined the
  eight observed age-abundance data points, the eight observed interquartile ranges for the age-abundance data points, and the eight observed
  age-prevalence data points into a vector S$_\text{obs}$ = [age-intensity$_1$,
  age-intensity$_2$, \ldots{}, age-intensity\_8, iqr$_1$, iqr$_2$, \ldots{}, iqr$_8$, age-prev$_1$,
  age-prev$_2$, \ldots{}, age-prev$_8$]. We did the same thing for all
  10,000 simulated age-abundance/prevalence profiles such that we had a
  10,000 x 24 matrix S$_\text{sim}$ where the columns matched the 24 dimensions
  in S$_\text{obs}$ and the rows corresponded to one of the randomly drawn
  particles.
\item
  To put the abundance measures and the prevalence data on the same
  scale, we standardized each column in S$_\text{sim}$ by subtracting the mean
  and dividing by the standard deviation of each column. We then
  performed this identical transformation on S$_\text{obs}$ \emph{based on the
  column-wise means and standard deviations from S$_\text{sim}$}. This
  transformation of S$_\text{obs}$ put the values of S$_\text{obs}$ in terms of
  deviations relative to the mean of the simulated data.
\item
  We then calculated the Euclidean distance between each row in S$_\text{sim}$
  and S$_\text{obs}$ (i.e. the L2 norm), which resulted in 10,000 distance
  measures. We then accepted the 500 particles that resulted in the
  500 smallest distances.
\item
  We then equally weighted each of these 500 accepted particles and
  resampled them with replacement 10,000 times. The upon sampling a
  particle, we perturbed each parameter in a particle using
 $\sigma_i$ $\times$ $Uniform(-1, 1)$ where sigma is the standard
  deviation of the ith parameter in the 500 accepted parameters. This
  perturbation helps the algorithm explore nearby parameter space.
\item
  We then repeated 3 - 9 with the 10,000 perturbed parameters 5
  times with one important change to step 9 after the first round of
  sampling. After identifying 500 accepted particles we used the
  following function to weight each particle: X. This is the
  importance weighting of a particle that accounts for the fact that
  particles are no longer being sampled from the prior distribution \citep{Beaumont2010}
  (DOUBLE CHECK WORDING).
\item
  After the fifth iteration {[}check{]}, the particles converged on a
  posterior distribution {[}How did we determine this\ldots{}deviation
  in posterior standard deviation and mean{]}.
\end{enumerate}

This algorithm provided posterior distributions for the five unknown
transmission and susceptibility parameters in the raccoon-\emph{B.
procyonis} model using commonly obtained age-abundance/prevalence
profiles.  

Before applying this algorithm to the empirical raccoon-\emph{B. procyonis} data, we tested the ability of this algorithm to recover known transmission parameters from age-intensity/prevalence data.  To do this, we fixed the four parameters of interest at specific, known values and simulated the raccoon IBM for 100 years (Supplementary Material).  We then sampled 189 raccoons present over the last two years of the model with the same age-structure as given in \cite{Weinstein2016} and considered this our ``observed \emph{in silico}'' data.  We then ran the ABC-SMC algorithm using this ``observed \emph{in silico}'' data and found that we could recover the true mean number of eggs encountered, the variability in eggs encountered, and the probability of encountering a rodent with some specificity, but could not reliably recover the egg contact parameter [WC] (see Supplementary Material). [Post-hoc simulation analyses illustrated that this parameter has little influence on the resulting age-intensity/age-prevalence profiles, which is consistent with the ABC-SMC algorithm not being able to identify it (Supplementary Material).] With this caveat in mind, we applied the ABC-SMC algorithm to the observed age-intensity/prevalence data from \cite{Weinstein2016}.

\subsection{Management Strategies for \emph{B. procyonis}}

We used our fully parameterized model to explore of how different
management strategies affected the dynamics of worm and raccoon
populations and human risk of encounter with \emph{B. procyonis}. We considered four
distinct management strategies in addition to combinations of these
management strategies: culling raccoons, birth control of raccoons,
anti-helmenthic baiting, and latrine cleanup. For each of these
management strategies, we considered different levels of management
effort (see below for definitions) and explored how different management
strategy-by-effort combinations affected four major predictions of the
model: 

\begin{enumerate}
    \item The total raccoon population 
    \item The total worm population
    \item The level of human risk [define above]
    \item The mean raccoon worm load
\end{enumerate}

For each management strategy, the model was run for 200 months to
obtain equilibrium dynamics, at which time the management strategy was
enacted monthly for an additional 200 months (FIG. X). Upon completion
of the simulation, the four major model predictions listed above were
computed based on the mean [max, min, var, too] over the last 24
months of the simulations.

The different management strategies were implemented as follows

\bigskip
\noindent
\emph{Culling raccoons}

[SARA: Could you add a paragraph on culling?]

To implement raccoon culling, we assumed that some number of traps were
randomly distributed in the raccoon world each month. We assumed that
per trap probability of catching a raccoon increased with raccoon
density following a type I functional response: $1 - \exp(-\text{raccoon
density} \gamma)$. $\gamma$ was estimated
based raccoon trapping data from  Prange et al. (2003) and Graser et al.
(2012) {[}probably more description here\ldots{}{]}. Finally, we assumed
that the number of raccoons were caught each month followed a
num\_trapped \textasciitilde{} Binomal(N, p) where the p = probability
of success was the per trap probability and the N = number of trials was
the number traps deployed that month. Because we assumed that all
individual raccoons were equally likely to be trapped, we then randomly
selected min(num\_trapped, total population size) raccoons from the
population to cull in that time step.

We also allowed trapping to occur in specific zones of human overlap and
only for particular age raccoons. When trapping in a particular zone(s)
of human overlap, trapping probability was determined only by the
density of raccoons in that zone(s) and only raccoons in that zone could
be trapped and culled. When trapping for a particular target age class,
trapping proceeded as described above, but raccoons that were not in the
target age class were released.

\bigskip
\noindent
\emph{Birth Control}

[SARA: Could you add a few sentences on birth control]

Birth control followed the exact same trapping regime as culling.
However, upon trapping a raccoon its reproductive ability was set to
zero for the remainder of its life {[}TODO: define reproductive ability
above{]}

\bigskip
\noindent
\emph{Anti-helminthic baiting}

[SARA: Could you add some information on anti-helminthic baiting]

We implemented anti-helminthic baiting by first randomly distributing
some number of baits into the raccoon world. Of the initial number of
baits distributed into the world {[}WC{]}, on average 60\% of these
baits were degraded or consumed by animals other than raccoons. For the
remaining baits, we assumed that all raccoons were equally likely to
consume a bait and randomly assigned baits to raccoons in the world. A
given raccoon could consume multiple baits in a month. Upon consuming a
bait, all the worms in a raccoon were killed, but the raccoon could
immediately start acquiring worms again from the environment and
rodents. Consuming one bait or multiple baits in a month had identical
effects on raccoon worm loads. Similar to culling and birth control, we
also allowed for anti-helminthic baiting in specific zones of human
overlap.

\emph{Latrine cleanup}

[SARA: Could you add dome background information]

The final management strategy we explored was latrine cleanup. {[}Define
latrine here or above?{]}. This strategy was implemented reducing or
eliminating the environmental egg pool each month. Once again, this
strategy could be specifically implemented in specific zones of human
overlap.

\section{Results}

\emph{Fitting transmission parameters with age-abundance/prevalence
profiles}

Figure in this section

\begin{itemize}
\item
  Plots of posterior distributions of parameters transmission parameters
\item
  Plots of observed and predicted age-abundance/prevalence profiles
\item
  Plots of model dynamics without any management after fitting
\end{itemize}

\emph{Management strategies for reducing} \emph{human risk of} B.
procyonis \emph{infection}

Figures in this section

\begin{itemize}
\item
  Heat maps of various strategies
\item
  Heat maps of strategy combinations
\end{itemize}

\section{Discussion}

When we talk about birth control- There is currently no approved raccoon
contraceptive (), although products developed for dogs and cats could be
optimized for raccoons given high cost and not great results, strategy
doesn't seem worth pursuing.

Culling- potential for a ``Hydra effect'': It's got a name! Abrams 2009.
When does greater mortality increase population size? the long history
and diverse mechanisms underlying the hydra effect. Ecology Letters
12:462--474 (named for greek myth, where chopping of one hydra head
caused two to grow back. While density dependent birth rates set up
possibility, and do lead to increased prev/intensity in animals, overall
reduction in rac population still reduces human risk, although not as
quickly perhaps as would if no ``hydra effect'' occurred.

\section{Acknowledgments} 

Acknowledgments, including funding
information, should appear in a brief statement at the end of the body
of the text. Acknowledgments of specific author contributions to the
paper should appear here.

% \textbf{Literature cited}

% Braae, U. C., B. Devleesschauwer, S. Gabriël, P. Dorny, N. Speybroeck,
% P. Magnussen, P. Torgerson, and M. V. Johansen. 2016. CystiSim -- An
% agent-based model for \emph{Taenia solium} transmission and control.
% PLoS Negl Trop Dis \textbf{10:} e0005184.

% Choisy, M., and P. Rohani. 2006. Harvesting can increase severity of
% wildlife disease epidemics. Proceedings of the Royal Society B:
% Biological Sciences \textbf{273:} 2025-2034.

% Elmore, S. A., R. B. Chipman, D. Slate, K. P. Huyvaert, K. C.
% VerCauteren, and A. T. Gilbert. 2017. Management and modeling approaches
% for controlling raccoon rabies: The road to elimination. PLoS Negl Trop
% Dis \textbf{11:} e0005249.

% Graeff-Teixeira, C., A. L. Morassutti, and K. R. Kazacos. 2016. Update
% on baylisascariasis, a highly pathogenic zoonotic infection. Clinical
% Microbiology Reviews \textbf{29:} 375-399.

% Kazacos, K. R. 2001. \emph{Baylisascaris procyonis} and related species.
% \emph{In} Parasitic Diseases of Wild Mammals W. M. Samuel, M. J. Pybus,
% and A. A. Kocan (eds.). Iowa State University Press, Ames, Iowa, p.
% 301-341.

% Kazacos, K. R. 2016. \emph{Baylisascaris} larva migrans. Vol. 1412 p.
% 1-122. U.S. Geological Survey Circular: 1412, Reston, Viriginia.

% Morters, M. K., O. Restif, K. Hampson, S. Cleaveland, J. L. N. Wood, and
% A. J. K. Conlan. 2013. Evidence-based control of canine rabies: a
% critical review of population density reduction. Journal of Animal
% Ecology \textbf{82:} 6-14.

% Page, K., J. C. Beasley, Z. H. Olson, T. J. Smyser, M. Downey, K. F.
% Kellner, S. E. McCord, T. S. Egan, 2nd, and O. E. Rhodes, Jr. 2011.
% Reducing \emph{Baylisascaris procyonis} roundworm larvae in raccoon
% latrines. Emerging Infectious Diseases \textbf{17:} 90-93.

% Page, K., T. J. Smyser, E. Dunkerton, E. Gavard, B. Larkin, and S.
% Gehrt. 2014. Reduction of \emph{Baylisascaris procyonis} eggs in raccoon
% latrines, suburban Chicago, Illinois, USA. Emerging Infectious Diseases
% \textbf{20:} 2137-2140.

% Page, K. L. 2013. Parasites and the conservation of small populations:
% The case of Baylisascaris procyonis. International Journal for
% Parasitology: Parasites and Wildlife \textbf{2:} 203-210.

% Rosatte, R. C., M. J. Pybus, and J. R. Gunson. 1986. Population
% reduction as a factor in the control of skunk rabies in Alberta. Journal
% of Wildlife Diseases \textbf{22:} 459-467.

% Sircar, A. D., F. Abanyie, D. Blumberg, P. Chin-Hong, K. S. Coulter, D.
% Cunningham, W. C. Huskins, C. Langelier, M. Reid, B. J. Scott, D.-A.
% Shirley, J. M. Babik, A. Belova, S. G. H. Sapp, I. McAuliffe, H. N.
% Rivera, M. J. Yabsley, and S. P. Montgomery. 2016. Raccoon roundworm
% infection associated with central nervous system disease and ocular
% disease---six states, 2013--2015. Morbidity and Mortality Weekly Report
% \textbf{65:} 930-933.

% Smyser, T. J., S. R. Johnson, M. D. Stallard, A. K. McGrew, L. K. Page,
% N. Crider, L. R. Ballweber, R. K. Swihart, and K. C. VerCauteren. 2015.
% Evaluation of anthelmintic fishmeal polymer baits for the control of
% \emph{Baylisascaris procyonis} in free-ranging raccoons (\emph{Procyon
% lotor}). Journal of Wildlife Diseases \textbf{51:} 640-650.

% Sterner, R. T., M. I. Meltzer, S. A. Shwiff, and D. Slate. 2009. Tactics
% and economics of wildlife oral rabies vaccination, Canada and the United
% States. Emerging Infectious Diseases \textbf{15:} 1176.

% Woodroffe, R., C. A. Donnelly, H. E. Jenkins, W. T. Johnston, D. R. Cox,
% F. J. Bourne, C. L. Cheeseman, R. J. Delahay, R. S. Clifton-Hadley, G.
% Gettinby, P. Gilks, R. G. Hewinson, J. P. McInerney, and W. I. Morrison.
% 2006. Culling and cattle controls influence tuberculosis risk for
% badgers. Proceedings of the National Academy of Sciences \textbf{103:}
% 14713-14717.

% \begin{longtable}[]{@{}llll@{}}
% \toprule
% \textbf{Submodel/ parameter description} & \textbf{Function/Value} &
% \textbf{Method or Source} &\tabularnewline
% \midrule
% \endhead
% \emph{Parasite-induced host mortality}: Determines the load-dependent
% probability of raccoon death over a month. & =
% \(1 - \frac{e^{\beta + \alpha*log(load + 1)}}{1 + e^{\beta + \alpha*log(load + 1)}}\)
% & &\tabularnewline
% Pathogenicity & & &\tabularnewline
% Baby death & & &\tabularnewline
% Random death probability & & &\tabularnewline
% Old death & & &\tabularnewline
% Age egg resistance & & &\tabularnewline
% Age susceptibility & & &\tabularnewline
% Rodent encounter probability & & &\tabularnewline
% Mouse worm mean & & &\tabularnewline
% Larval worm infectivity & & &\tabularnewline
% First reproduction age & & &\tabularnewline
% Litter size & & &\tabularnewline
% Encounter mean & & &\tabularnewline
% Encounter k & & &\tabularnewline
% Egg contact & & &\tabularnewline
% Infectivity & & &\tabularnewline
% Resistance & & &\tabularnewline
% Egg decay & & &\tabularnewline
% Worm survival threshold & & &\tabularnewline
% Worm survival slope & & &\tabularnewline
% \bottomrule
% \end{longtable}

\textbf{Tables} (1 per page)

\textbf{Figure Legends}

\textbf{Figures} (1 per page, labeled ``Figure 1'', etc)

\begin{itemize}
\item
  Figure sizes should be no more than 6 inches wide and 7 inches high.
  When possible, submit figures in the size you wish to have them appear
  in the journal. Most illustrations, except some maps and very wide
  graphs, should be 1-column width (3 inches) at a resolution of 600
  dpi.
\item
  The font size of the \emph{x}- and \emph{y}-axis numbers should be
  slightly smaller than the axis label. A consistent font (Helvetica is
  preferred) should be used throughout. Use boldface type only if
  required for journal style. Use sentence case (i.e., only capitalize
  the first word) for axis titles, labels, and legends.
\item
  For symbols and lines, avoid very small sizes and line thicknesses (1
  point width stroke or greater is preferable). All elements of a figure
  should appear with the same degree of intensity.~ If different degrees
  of intensity need to be conveyed, lines should differ by 1 point width
  for clarity.
\end{itemize}

\textbf{Appendices}

\bibliographystyle{/Users/mqwilber/Dropbox/Documents/Bibformats/ecology.bst}
\bibliography{/Users/mqwilber/Dropbox/Documents/Bibfiles/Projects_and_Permits-raccoon_IBM.bib}


\end{document}